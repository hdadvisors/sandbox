% Options for packages loaded elsewhere
\PassOptionsToPackage{unicode}{hyperref}
\PassOptionsToPackage{hyphens}{url}
\PassOptionsToPackage{dvipsnames,svgnames,x11names}{xcolor}
%
\documentclass[
  10pt,
  letterpaper,
  DIV=11,
  numbers=noendperiod]{scrartcl}

\usepackage{amsmath,amssymb}
\usepackage{iftex}
\ifPDFTeX
  \usepackage[T1]{fontenc}
  \usepackage[utf8]{inputenc}
  \usepackage{textcomp} % provide euro and other symbols
\else % if luatex or xetex
  \usepackage{unicode-math}
  \defaultfontfeatures{Scale=MatchLowercase}
  \defaultfontfeatures[\rmfamily]{Ligatures=TeX,Scale=1}
\fi
\usepackage{lmodern}
\ifPDFTeX\else  
    % xetex/luatex font selection
  \setmainfont[]{Arial}
\fi
% Use upquote if available, for straight quotes in verbatim environments
\IfFileExists{upquote.sty}{\usepackage{upquote}}{}
\IfFileExists{microtype.sty}{% use microtype if available
  \usepackage[]{microtype}
  \UseMicrotypeSet[protrusion]{basicmath} % disable protrusion for tt fonts
}{}
\makeatletter
\@ifundefined{KOMAClassName}{% if non-KOMA class
  \IfFileExists{parskip.sty}{%
    \usepackage{parskip}
  }{% else
    \setlength{\parindent}{0pt}
    \setlength{\parskip}{6pt plus 2pt minus 1pt}}
}{% if KOMA class
  \KOMAoptions{parskip=half}}
\makeatother
\usepackage{xcolor}
\usepackage[footskip=5mm,top=20mm,bottom=25mm,left=25mm,right=25mm]{geometry}
\setlength{\emergencystretch}{3em} % prevent overfull lines
\setcounter{secnumdepth}{1}
% Make \paragraph and \subparagraph free-standing
\ifx\paragraph\undefined\else
  \let\oldparagraph\paragraph
  \renewcommand{\paragraph}[1]{\oldparagraph{#1}\mbox{}}
\fi
\ifx\subparagraph\undefined\else
  \let\oldsubparagraph\subparagraph
  \renewcommand{\subparagraph}[1]{\oldsubparagraph{#1}\mbox{}}
\fi

\usepackage{color}
\usepackage{fancyvrb}
\newcommand{\VerbBar}{|}
\newcommand{\VERB}{\Verb[commandchars=\\\{\}]}
\DefineVerbatimEnvironment{Highlighting}{Verbatim}{commandchars=\\\{\}}
% Add ',fontsize=\small' for more characters per line
\usepackage{framed}
\definecolor{shadecolor}{RGB}{241,243,245}
\newenvironment{Shaded}{\begin{snugshade}}{\end{snugshade}}
\newcommand{\AlertTok}[1]{\textcolor[rgb]{0.68,0.00,0.00}{#1}}
\newcommand{\AnnotationTok}[1]{\textcolor[rgb]{0.37,0.37,0.37}{#1}}
\newcommand{\AttributeTok}[1]{\textcolor[rgb]{0.40,0.45,0.13}{#1}}
\newcommand{\BaseNTok}[1]{\textcolor[rgb]{0.68,0.00,0.00}{#1}}
\newcommand{\BuiltInTok}[1]{\textcolor[rgb]{0.00,0.23,0.31}{#1}}
\newcommand{\CharTok}[1]{\textcolor[rgb]{0.13,0.47,0.30}{#1}}
\newcommand{\CommentTok}[1]{\textcolor[rgb]{0.37,0.37,0.37}{#1}}
\newcommand{\CommentVarTok}[1]{\textcolor[rgb]{0.37,0.37,0.37}{\textit{#1}}}
\newcommand{\ConstantTok}[1]{\textcolor[rgb]{0.56,0.35,0.01}{#1}}
\newcommand{\ControlFlowTok}[1]{\textcolor[rgb]{0.00,0.23,0.31}{#1}}
\newcommand{\DataTypeTok}[1]{\textcolor[rgb]{0.68,0.00,0.00}{#1}}
\newcommand{\DecValTok}[1]{\textcolor[rgb]{0.68,0.00,0.00}{#1}}
\newcommand{\DocumentationTok}[1]{\textcolor[rgb]{0.37,0.37,0.37}{\textit{#1}}}
\newcommand{\ErrorTok}[1]{\textcolor[rgb]{0.68,0.00,0.00}{#1}}
\newcommand{\ExtensionTok}[1]{\textcolor[rgb]{0.00,0.23,0.31}{#1}}
\newcommand{\FloatTok}[1]{\textcolor[rgb]{0.68,0.00,0.00}{#1}}
\newcommand{\FunctionTok}[1]{\textcolor[rgb]{0.28,0.35,0.67}{#1}}
\newcommand{\ImportTok}[1]{\textcolor[rgb]{0.00,0.46,0.62}{#1}}
\newcommand{\InformationTok}[1]{\textcolor[rgb]{0.37,0.37,0.37}{#1}}
\newcommand{\KeywordTok}[1]{\textcolor[rgb]{0.00,0.23,0.31}{#1}}
\newcommand{\NormalTok}[1]{\textcolor[rgb]{0.00,0.23,0.31}{#1}}
\newcommand{\OperatorTok}[1]{\textcolor[rgb]{0.37,0.37,0.37}{#1}}
\newcommand{\OtherTok}[1]{\textcolor[rgb]{0.00,0.23,0.31}{#1}}
\newcommand{\PreprocessorTok}[1]{\textcolor[rgb]{0.68,0.00,0.00}{#1}}
\newcommand{\RegionMarkerTok}[1]{\textcolor[rgb]{0.00,0.23,0.31}{#1}}
\newcommand{\SpecialCharTok}[1]{\textcolor[rgb]{0.37,0.37,0.37}{#1}}
\newcommand{\SpecialStringTok}[1]{\textcolor[rgb]{0.13,0.47,0.30}{#1}}
\newcommand{\StringTok}[1]{\textcolor[rgb]{0.13,0.47,0.30}{#1}}
\newcommand{\VariableTok}[1]{\textcolor[rgb]{0.07,0.07,0.07}{#1}}
\newcommand{\VerbatimStringTok}[1]{\textcolor[rgb]{0.13,0.47,0.30}{#1}}
\newcommand{\WarningTok}[1]{\textcolor[rgb]{0.37,0.37,0.37}{\textit{#1}}}

\providecommand{\tightlist}{%
  \setlength{\itemsep}{0pt}\setlength{\parskip}{0pt}}\usepackage{longtable,booktabs,array}
\usepackage{calc} % for calculating minipage widths
% Correct order of tables after \paragraph or \subparagraph
\usepackage{etoolbox}
\makeatletter
\patchcmd\longtable{\par}{\if@noskipsec\mbox{}\fi\par}{}{}
\makeatother
% Allow footnotes in longtable head/foot
\IfFileExists{footnotehyper.sty}{\usepackage{footnotehyper}}{\usepackage{footnote}}
\makesavenoteenv{longtable}
\usepackage{graphicx}
\makeatletter
\def\maxwidth{\ifdim\Gin@nat@width>\linewidth\linewidth\else\Gin@nat@width\fi}
\def\maxheight{\ifdim\Gin@nat@height>\textheight\textheight\else\Gin@nat@height\fi}
\makeatother
% Scale images if necessary, so that they will not overflow the page
% margins by default, and it is still possible to overwrite the defaults
% using explicit options in \includegraphics[width, height, ...]{}
\setkeys{Gin}{width=\maxwidth,height=\maxheight,keepaspectratio}
% Set default figure placement to htbp
\makeatletter
\def\fps@figure{htbp}
\makeatother

\usepackage{booktabs}
\usepackage{longtable}
\usepackage{array}
\usepackage{multirow}
\usepackage{wrapfig}
\usepackage{float}
\usepackage{colortbl}
\usepackage{pdflscape}
\usepackage{tabu}
\usepackage{threeparttable}
\usepackage{threeparttablex}
\usepackage[normalem]{ulem}
\usepackage{makecell}
\usepackage{xcolor}
\renewcommand{\thepage}{A7-\arabic{page}}
\renewcommand{\thesection}{A7.\arabic{section}}
\KOMAoption{captions}{tableheading}
\makeatletter
\@ifpackageloaded{tcolorbox}{}{\usepackage[skins,breakable]{tcolorbox}}
\@ifpackageloaded{fontawesome5}{}{\usepackage{fontawesome5}}
\definecolor{quarto-callout-color}{HTML}{909090}
\definecolor{quarto-callout-note-color}{HTML}{0758E5}
\definecolor{quarto-callout-important-color}{HTML}{CC1914}
\definecolor{quarto-callout-warning-color}{HTML}{EB9113}
\definecolor{quarto-callout-tip-color}{HTML}{00A047}
\definecolor{quarto-callout-caution-color}{HTML}{FC5300}
\definecolor{quarto-callout-color-frame}{HTML}{acacac}
\definecolor{quarto-callout-note-color-frame}{HTML}{4582ec}
\definecolor{quarto-callout-important-color-frame}{HTML}{d9534f}
\definecolor{quarto-callout-warning-color-frame}{HTML}{f0ad4e}
\definecolor{quarto-callout-tip-color-frame}{HTML}{02b875}
\definecolor{quarto-callout-caution-color-frame}{HTML}{fd7e14}
\makeatother
\makeatletter
\@ifpackageloaded{caption}{}{\usepackage{caption}}
\AtBeginDocument{%
\ifdefined\contentsname
  \renewcommand*\contentsname{Table of contents}
\else
  \newcommand\contentsname{Table of contents}
\fi
\ifdefined\listfigurename
  \renewcommand*\listfigurename{List of Figures}
\else
  \newcommand\listfigurename{List of Figures}
\fi
\ifdefined\listtablename
  \renewcommand*\listtablename{List of Tables}
\else
  \newcommand\listtablename{List of Tables}
\fi
\ifdefined\figurename
  \renewcommand*\figurename{Figure}
\else
  \newcommand\figurename{Figure}
\fi
\ifdefined\tablename
  \renewcommand*\tablename{Table}
\else
  \newcommand\tablename{Table}
\fi
}
\@ifpackageloaded{float}{}{\usepackage{float}}
\floatstyle{ruled}
\@ifundefined{c@chapter}{\newfloat{codelisting}{h}{lop}}{\newfloat{codelisting}{h}{lop}[chapter]}
\floatname{codelisting}{Listing}
\newcommand*\listoflistings{\listof{codelisting}{List of Listings}}
\makeatother
\makeatletter
\makeatother
\makeatletter
\@ifpackageloaded{caption}{}{\usepackage{caption}}
\@ifpackageloaded{subcaption}{}{\usepackage{subcaption}}
\makeatother
\ifLuaTeX
  \usepackage{selnolig}  % disable illegal ligatures
\fi
\usepackage{bookmark}

\IfFileExists{xurl.sty}{\usepackage{xurl}}{} % add URL line breaks if available
\urlstyle{same} % disable monospaced font for URLs
\hypersetup{
  pdftitle={Appendix 7: Scenario Models Methodology},
  colorlinks=true,
  linkcolor={blue},
  filecolor={Maroon},
  citecolor={Blue},
  urlcolor={Blue},
  pdfcreator={LaTeX via pandoc}}

\title{Appendix 7: Scenario Models Methodology}
\author{}
\date{}

\begin{document}
\maketitle

\vspace{-1.5cm}

This document provides a step-by-step methodology to model three
different Local Rent Supplement Program (LRSP) scenarios for the City of
Alexandria. Data is created, transformed, and visualized using the R
coding language. The R libraries used for this analysis are listed
below.

\begin{Shaded}
\begin{Highlighting}[]
\FunctionTok{library}\NormalTok{(tidyverse)}
\FunctionTok{library}\NormalTok{(scales)}
\FunctionTok{library}\NormalTok{(kableExtra)}
\FunctionTok{library}\NormalTok{(formattable)}
\FunctionTok{library}\NormalTok{(gt)}
\FunctionTok{library}\NormalTok{(ggtext)}
\FunctionTok{library}\NormalTok{(janitor)}
\end{Highlighting}
\end{Shaded}

\section{Standard parameters}\label{standard-parameters}

\subsection{Income limits}\label{income-limits}

The models use HUD's FY 2023 Multifamily Tax Subsidy Projects (MTSP)
Income Limits for Washington-Arlington-Alexandria, DC-VA-MD HUD Metro
FMR Area.\footnote{\href{https://www.huduser.gov/portal/datasets/il/il2023/2023sum_mtsp.odn?inputname=METRO47900M47900*Washington-Arlington-Alexandria\%2C+DC-VA-MD+HUD+Metro+FMR+Area&area_choice=hmfa&year=2023\#top}{FY
  2023 MTSP Income Limits} (Accessed 2024-01-19)} These are the official
income limits used to determine eligibility for LIHTC projects and other
affordable multifamily properties financed by tax-exempt bonds.

While the MTSP limits differ slightly from the standard income limits
used for Housing Choice Vouchers, public housing, and other
HUD-supported assistance programs, they are used here because they
publish limits for a greater range of AMI levels, including 40\% AMI and
60\% AMI. The City of Alexandria generally uses MTSP limits for its
housing programs.

\begin{table}[H]

\caption{\label{tbl-hud-ami}FY 2023 MTSP Income Limits for
Washington-Arlington-Alexandria, DC-VA-MD HUD Metro FMR Area}

\centering{

\begin{tabu} to \linewidth {>{\raggedright}X>{\centering}X>{\centering}X>{\centering}X>{\centering}X>{\centering}X>{\centering}X}
\toprule
\textbf{AMI} & \textbf{1 person} & \textbf{2 person} & \textbf{3 person} & \textbf{4 person} & \textbf{5 person} & \textbf{6 person}\\
\midrule
20\% AMI & \$21,100 & \$24,120 & \$27,140 & \$30,140 & \$32,460 & \$34,980\\
30\% AMI & \$31,650 & \$36,180 & \$40,710 & \$45,210 & \$48,840 & \$52,470\\
40\% AMI & \$42,200 & \$48,240 & \$54,280 & \$60,280 & \$65,210 & \$69,960\\
50\% AMI & \$52,750 & \$60,300 & \$67,850 & \$75,350 & \$81,400 & \$87,450\\
60\% AMI & \$63,300 & \$72,660 & \$81,420 & \$90,240 & \$97,460 & \$104,940\\
70\% AMI & \$73,850 & \$84,240 & \$94,990 & \$105,490 & \$113,960 & \$122,430\\
80\% AMI & \$84,400 & \$96,480 & \$108,560 & \$120,560 & \$130,240 & \$139,920\\
\bottomrule
\end{tabu}

}

\end{table}%

\newpage

\subsection{Fair Market Rents}\label{fair-market-rents}

Models where the rent subsidy is calculated based on Fair Market Rents
(FMR) use the current Small Area Fair Market Rents (SAFMR) adopted by
the Alexandria Redevelopment and Housing Authority for 2023. SAFMRs are
provided by ZIP code.

While actual subsidy amounts will depend on the ZIP code where the
tenant lives, models will use the average values (by unit size) across
all ZIP codes. This is a simplification to avoid making assumptions
about the geographic distribution of participating households.

\begin{table}[H]

\caption{\label{tbl-fmrs}ARHA 2023 Payment Standards}

\centering{

\begin{tabu} to \linewidth {>{\raggedright}X>{\centering}X>{\centering}X>{\centering}X>{\centering}X>{\centering}X}
\toprule
\textbf{ZIP code} & \textbf{Studio} & \textbf{1 bedroom} & \textbf{2 bedroom} & \textbf{3 bedroom} & \textbf{4 bedroom}\\
\midrule
22301 & \$2,013 & \$2,046 & \$2,332 & \$2,915 & \$3,476\\
22302 & \$1,980 & \$2,013 & \$2,288 & \$2,860 & \$3,410\\
22304 & \$1,914 & \$1,947 & \$2,211 & \$2,761 & \$3,300\\
22305 & \$1,859 & \$1,892 & \$2,156 & \$2,695 & \$3,212\\
22311 & \$1,936 & \$1,969 & \$2,244 & \$2,805 & \$3,344\\
22312 & \$1,848 & \$1,870 & \$2,134 & \$2,673 & \$3,179\\
22313 & \$1,782 & \$1,815 & \$2,068 & \$2,585 & \$3,080\\
22314 & \$2,563 & \$2,607 & \$2,970 & \$3,718 & \$4,433\\
\textbf{Average} & \textbf{\$1,986.88} & \textbf{\$2,019.88} & \textbf{\$2,300.38} & \textbf{\$2,876.50} & \textbf{\$3,429.25}\\
\bottomrule
\end{tabu}

}

\end{table}%

\newpage

\section{Scenario A - Reduce Cost Burden for 30\% to 50\% AMI
Households}\label{scenario-a---reduce-cost-burden-for-30-to-50-ami-households}

This scenario outlines a LRSP with a total annual allocation of
\$500,000. The primary goal of the program is to reduce housing cost
burden among households with incomes between 30\% and 50\% AMI. The
model uses the following inputs to estimate the number of households
served.

\begin{longtable}[]{@{}
  >{\raggedright\arraybackslash}p{(\columnwidth - 2\tabcolsep) * \real{0.4022}}
  >{\raggedright\arraybackslash}p{(\columnwidth - 2\tabcolsep) * \real{0.5978}}@{}}
\toprule\noalign{}
\begin{minipage}[b]{\linewidth}\raggedright
\textbf{Variable}
\end{minipage} & \begin{minipage}[b]{\linewidth}\raggedright
\textbf{Input}
\end{minipage} \\
\midrule\noalign{}
\endhead
\bottomrule\noalign{}
\endlastfoot
\emph{Total program budget} & \$500,000\newline \\
\emph{Eligibility} & Household income between\newline
30\% and 50\% AMI\newline \\
\emph{Subsidy amount} & Difference between the affordable
monthly\newline rent at 60\% AMI and the affordable monthly\newline rent
at 40\% AMI\newline \\
\begin{minipage}[t]{\linewidth}\raggedright
\emph{Distribution of household}\\
\emph{sizes among participants}\strut
\end{minipage} & 15\% - 1-person\newline
15\% - 2-person\newline
20\% - 3-person\newline
20\% - 4-person\newline
20\% - 5-person\newline
10\% - 6-person\newline \\
\emph{Administrative overhead} & 15\% of total program budget \\
\end{longtable}

Notes:

\begin{itemize}
\tightlist
\item
  No other eligibility conditions apply.
\item
  ``Affordable monthly rent'' is 30\% of gross household income.
\item
  The subsidy calculated for each household is respective to their
  household size. No assumed breakdown of households by AMI is needed.
\item
  The administrative overhead includes housing-specific case management.
\end{itemize}

\newpage

\subsection{Inputs}\label{inputs}

Assign budget (dollars) and overhead costs (percent) variables:

\begin{Shaded}
\begin{Highlighting}[]
\CommentTok{\# Budget allocation}
\NormalTok{sA\_budget }\OtherTok{\textless{}{-}} \DecValTok{500000}

\CommentTok{\# Overhead percentage}
\NormalTok{sA\_overhead }\OtherTok{\textless{}{-}} \FloatTok{0.15}
\end{Highlighting}
\end{Shaded}

Assign household distributions by household size (number of persons):

\begin{Shaded}
\begin{Highlighting}[]
\CommentTok{\# Distribution of households by household size}
\NormalTok{sA\_person }\OtherTok{\textless{}{-}} \FunctionTok{tibble}\NormalTok{(}
  \AttributeTok{hh\_size =} \FunctionTok{paste0}\NormalTok{(}\StringTok{"person"}\NormalTok{, }\DecValTok{1}\SpecialCharTok{:}\DecValTok{6}\NormalTok{),}
  \AttributeTok{pct =} \FunctionTok{c}\NormalTok{(}\FloatTok{0.15}\NormalTok{, }\FloatTok{0.15}\NormalTok{, }\FloatTok{0.20}\NormalTok{, }\FloatTok{0.20}\NormalTok{, }\FloatTok{0.20}\NormalTok{, }\FloatTok{0.10}\NormalTok{)}
\NormalTok{)}
\end{Highlighting}
\end{Shaded}

\subsection{Calculations}\label{calculations}

Calculate affordable rents at 40\% AMI and 60\% AMI for households with
1 to 6 persons (\texttt{hh\_size}) to determine monthly subsidy amounts
(\texttt{subsidy}):

\begin{Shaded}
\begin{Highlighting}[]
\CommentTok{\# Monthly subsidy about by household size}
\NormalTok{sA\_subsidy }\OtherTok{\textless{}{-}}\NormalTok{ hud\_ami }\SpecialCharTok{|\textgreater{}} 
  \FunctionTok{filter}\NormalTok{(}
\NormalTok{    AMI }\SpecialCharTok{\%in\%} \FunctionTok{c}\NormalTok{(}\StringTok{"40\% AMI"}\NormalTok{, }\StringTok{"60\% AMI"}\NormalTok{), }\CommentTok{\# 40\% and 60\% AMI only}
    \FunctionTok{str\_detect}\NormalTok{(hh\_size, }\StringTok{"[123456]"}\NormalTok{)   }\CommentTok{\# 1{-}6 person households only}
\NormalTok{    ) }\SpecialCharTok{|\textgreater{}} 
  \FunctionTok{mutate}\NormalTok{(}
    \AttributeTok{aff\_rent =}\NormalTok{ income}\SpecialCharTok{/}\DecValTok{12} \SpecialCharTok{*} \FloatTok{0.3}        \CommentTok{\# 30\% of monthly income}
\NormalTok{  ) }\SpecialCharTok{|\textgreater{}} 
  \FunctionTok{select}\NormalTok{(}\SpecialCharTok{{-}}\DecValTok{3}\NormalTok{) }\SpecialCharTok{|\textgreater{}} 
  \FunctionTok{pivot\_wider}\NormalTok{(}
    \AttributeTok{names\_from =}\NormalTok{ AMI,}
    \AttributeTok{values\_from =}\NormalTok{ aff\_rent}
\NormalTok{  ) }\SpecialCharTok{|\textgreater{}} 
  \FunctionTok{mutate}\NormalTok{(}
    \AttributeTok{subsidy =} \StringTok{\textasciigrave{}}\AttributeTok{60\% AMI}\StringTok{\textasciigrave{}} \SpecialCharTok{{-}} \StringTok{\textasciigrave{}}\AttributeTok{40\% AMI}\StringTok{\textasciigrave{}}   \CommentTok{\# Calculate subsidy}
\NormalTok{  ) }\SpecialCharTok{|\textgreater{}} 
  \FunctionTok{select}\NormalTok{(}\DecValTok{1}\NormalTok{, }\DecValTok{4}\NormalTok{)}
\end{Highlighting}
\end{Shaded}

\begin{tabu} to \linewidth {>{\raggedright}X>{\centering}X}
\toprule
\textbf{hh\_size} & \textbf{subsidy}\\
\midrule
person1 & 527.50\\
person2 & 610.50\\
person3 & 678.50\\
person4 & 749.00\\
person5 & 806.25\\
person6 & 874.50\\
\bottomrule
\end{tabu}

\newpage

Join the monthly subsidy amounts by household size (\texttt{subsidy})
and calculate annual subsidy per household (\texttt{subsidy\_annual}):

\begin{Shaded}
\begin{Highlighting}[]
\CommentTok{\# Annual subsidy per household size}
\NormalTok{sA\_subsidy\_annual }\OtherTok{\textless{}{-}}\NormalTok{ sA\_person }\SpecialCharTok{|\textgreater{}}
  \FunctionTok{left\_join}\NormalTok{(sA\_subsidy) }\SpecialCharTok{|\textgreater{}} 
  \FunctionTok{mutate}\NormalTok{(}\AttributeTok{subsidy\_annual =}\NormalTok{ subsidy }\SpecialCharTok{*} \DecValTok{12}\NormalTok{)}
\end{Highlighting}
\end{Shaded}

\begin{tabu} to \linewidth {>{\raggedright}X>{\centering}X>{\centering}X>{\centering}X}
\toprule
\textbf{hh\_size} & \textbf{pct} & \textbf{subsidy} & \textbf{subsidy\_annual}\\
\midrule
person1 & 0.15 & 527.50 & 6330.00\\
person2 & 0.15 & 610.50 & 7326.00\\
person3 & 0.20 & 678.50 & 8142.00\\
person4 & 0.20 & 749.00 & 8988.00\\
person5 & 0.20 & 806.25 & 9675.00\\
person6 & 0.10 & 874.50 & 10494.00\\
\bottomrule
\end{tabu}

\hfill\break

Calculate the theoretical share of subsidy allocated for each household
size (\texttt{subsidy\_share}). Determine the number of households
served (\texttt{hh\_served}) by normalizing \texttt{subsidy\_share} to
the known budget, then calculate the budget share (\texttt{budget}) for
each household size:

\begin{Shaded}
\begin{Highlighting}[]
\CommentTok{\# Annual subsidy per household type}
\NormalTok{sA\_served }\OtherTok{\textless{}{-}}\NormalTok{ sA\_subsidy\_annual }\SpecialCharTok{|\textgreater{}}
  \FunctionTok{mutate}\NormalTok{(}
    \AttributeTok{subsidy\_share =}\NormalTok{ subsidy\_annual }\SpecialCharTok{*}\NormalTok{ pct, }\CommentTok{\# Subsidy per HH type}
    \AttributeTok{hh\_served =} \CommentTok{\# Adjust to known budget}
\NormalTok{      pct}\SpecialCharTok{*}\NormalTok{(}
\NormalTok{        sA\_budget }\SpecialCharTok{*}\NormalTok{ (}\DecValTok{1} \SpecialCharTok{{-}}\NormalTok{ sA\_overhead)}
\NormalTok{        )}\SpecialCharTok{/}\FunctionTok{sum}\NormalTok{(subsidy\_share) }
\NormalTok{  ) }\SpecialCharTok{|\textgreater{}} 
  \FunctionTok{mutate}\NormalTok{(}\AttributeTok{budget =}\NormalTok{ hh\_served }\SpecialCharTok{*}\NormalTok{ subsidy\_annual)}
\end{Highlighting}
\end{Shaded}

\begin{tabu} to \linewidth {>{\raggedright}X>{\centering}X>{\centering}X>{\centering}X>{\centering}X}
\toprule
\textbf{hh\_size} & \textbf{subsidy\_annual} & \textbf{subsidy\_share} & \textbf{budget} & \textbf{hh\_served}\\
\midrule
person1 & 6330.00 & 949.50 & 47706.23 & 7.537\\
person2 & 7326.00 & 1098.90 & 55212.62 & 7.537\\
person3 & 8142.00 & 1628.40 & 81816.57 & 10.049\\
person4 & 8988.00 & 1797.60 & 90317.78 & 10.049\\
person5 & 9675.00 & 1935.00 & 97221.24 & 10.049\\
person6 & 10494.00 & 1049.40 & 52725.56 & 5.024\\
Total & - & - & 425000.00 & 50.244\\
\bottomrule
\end{tabu}

\newpage

\subsection{Model results}\label{model-results}

Round each estimate to the nearest whole number and determine total:

\begin{Shaded}
\begin{Highlighting}[]
\CommentTok{\# Rounded estimates with grand total}
\NormalTok{sA\_estimate }\OtherTok{\textless{}{-}}\NormalTok{ sA\_served }\SpecialCharTok{|\textgreater{}}
  \FunctionTok{select}\NormalTok{(}\DecValTok{1}\NormalTok{, }\DecValTok{7}\NormalTok{, }\DecValTok{6}\NormalTok{) }\SpecialCharTok{|\textgreater{}} 
  \FunctionTok{mutate}\NormalTok{(}
    \AttributeTok{hh\_served =} \FunctionTok{round}\NormalTok{(hh\_served),}
    \AttributeTok{hh\_size =} \FunctionTok{case\_match}\NormalTok{(}
\NormalTok{      hh\_size,}
      \StringTok{"person1"} \SpecialCharTok{\textasciitilde{}} \StringTok{"1 person"}\NormalTok{,}
      \StringTok{"person2"} \SpecialCharTok{\textasciitilde{}} \StringTok{"2 person"}\NormalTok{,}
      \StringTok{"person3"} \SpecialCharTok{\textasciitilde{}} \StringTok{"3 person"}\NormalTok{,}
      \StringTok{"person4"} \SpecialCharTok{\textasciitilde{}} \StringTok{"4 person"}\NormalTok{,}
      \StringTok{"person5"} \SpecialCharTok{\textasciitilde{}} \StringTok{"5 person"}\NormalTok{,}
      \StringTok{"person6"} \SpecialCharTok{\textasciitilde{}} \StringTok{"6 person"}
\NormalTok{    )}
\NormalTok{  ) }\SpecialCharTok{|\textgreater{}} 
  \FunctionTok{adorn\_totals}\NormalTok{()}
\end{Highlighting}
\end{Shaded}

\begin{table}[H]

\caption{\label{tbl-sA}Scenario A - Estimated Households Served by
Household Size}

\centering{

\begin{tabu} to \linewidth {>{\raggedright}X>{\centering}X>{\centering}X}
\toprule
\textbf{Household size} & \textbf{Annual cost} & \textbf{Households served}\\
\midrule
1 person & \$47,706 & 8\\
2 person & \$55,213 & 8\\
3 person & \$81,817 & 10\\
4 person & \$90,318 & 10\\
5 person & \$97,221 & 10\\
6 person & \$52,726 & 5\\
\textbf{Total} & \textbf{\$425,000} & \textbf{51}\\
\bottomrule
\end{tabu}

}

\end{table}%

\hfill\break

\begin{tcolorbox}[enhanced jigsaw, breakable, toprule=.15mm, opacitybacktitle=0.6, colframe=quarto-callout-note-color-frame, colback=white, opacityback=0, coltitle=black, colbacktitle=quarto-callout-note-color!10!white, rightrule=.15mm, toptitle=1mm, titlerule=0mm, title=\textcolor{quarto-callout-note-color}{\faInfo}\hspace{0.5em}{Scenario A results}, bottomtitle=1mm, bottomrule=.15mm, leftrule=.75mm, arc=.35mm, left=2mm]

\emph{Average annual program cost per household: \$9,803.92}

\hfill\break
Under Scenario A, a total program budget of \$500,000 with a 15\%
administrative overhead leaves \$425,000 to fund rental assistance.
Given the assumed household distribution by household size, the total
number of households served is 51.

\end{tcolorbox}

\newpage

\section{Scenario B - Stabilize Unhoused
Persons}\label{scenario-b---stabilize-unhoused-persons}

This scenario outlines a LRSP serving a total of 150 households
experiencing housing insecurity. The primary goal of the program is to
provide deep rental assistance to help these households achieve housing
stability and avoid homelessness. The model uses the following inputs to
estimate the annual program cost required to serve 150 households.

\begin{longtable}[]{@{}
  >{\raggedright\arraybackslash}p{(\columnwidth - 2\tabcolsep) * \real{0.4022}}
  >{\raggedright\arraybackslash}p{(\columnwidth - 2\tabcolsep) * \real{0.5978}}@{}}
\toprule\noalign{}
\begin{minipage}[b]{\linewidth}\raggedright
\textbf{Variable}
\end{minipage} & \begin{minipage}[b]{\linewidth}\raggedright
\textbf{Input}
\end{minipage} \\
\midrule\noalign{}
\endhead
\bottomrule\noalign{}
\endlastfoot
\emph{Total households served} & 150\newline \\
\emph{Eligibility} & Household/individual determined to be\newline
homeless in City's annual Point-in-Time count\newline \\
\emph{Subsidy amount} & Difference between the affordable
monthly\newline rent at 60\% AMI and the households' current\newline
affordable monthly rent\newline \\
\begin{minipage}[t]{\linewidth}\raggedright
\emph{Distribution of household types}\\
\emph{among participants}\strut
\end{minipage} & 2/3 - Single-person\newline
1/3 - Household with children\newline \\
\begin{minipage}[t]{\linewidth}\raggedright
\emph{Distribution of unit sizes}\\
\emph{among participants}\strut
\end{minipage} & 2/3 - Studios\newline
1/3 - 2-bedroom\newline \\
\begin{minipage}[t]{\linewidth}\raggedright
\emph{Distribution of incomes}\\
\emph{among participants}\strut
\end{minipage} & 50\% - SSI income\newline
50\% - \$1,500 per month\newline \\
\emph{Administrative overhead} & 20\% of total program budget \\
\end{longtable}

Notes:

\begin{itemize}
\tightlist
\item
  No other eligibility conditions apply.
\item
  ``Affordable monthly rent'' is 30\% of gross household income.
\item
  The subsidy calculated for each household is respective to their
  household size.
\item
  The administrative overhead is higher than Scenario A to accommodate
  more intensive case management requirements for persons experiencing
  homelessness.
\end{itemize}

\newpage

\subsection{Inputs}\label{inputs-1}

Assign households served and overhead costs (percent) variables:

\begin{Shaded}
\begin{Highlighting}[]
\CommentTok{\# Total households served}
\NormalTok{sB\_hh\_served }\OtherTok{\textless{}{-}} \DecValTok{150}

\CommentTok{\# Overhead percentage}
\NormalTok{sB\_overhead }\OtherTok{\textless{}{-}} \FloatTok{0.20}
\end{Highlighting}
\end{Shaded}

We can reasonably assume that all single-person households will live in
studios, while all households with children will live in 2-bedroom
units. Therefore, we do not need separate distribution shares for each.
However, we do need to determine more specific household sizes.

For this model, among households with children, we will assume the
following breakdown, as shown in the code below:

\begin{itemize}
\tightlist
\item
  1/2 are 2-person (adult and child)
\item
  1/4 are 3-person (adult and two children, or two adults and child)
\item
  1/4 are 4-person (adult and three children, or two adults and two
  children)
\end{itemize}

\begin{Shaded}
\begin{Highlighting}[]
\CommentTok{\# Distribution of households by size}
\NormalTok{sB\_person }\OtherTok{\textless{}{-}} \FunctionTok{c}\NormalTok{(}
  \StringTok{\textasciigrave{}}\AttributeTok{person1}\StringTok{\textasciigrave{}} \OtherTok{=} \FloatTok{0.667}\NormalTok{, }\CommentTok{\# 2/3}
  \StringTok{\textasciigrave{}}\AttributeTok{person2}\StringTok{\textasciigrave{}} \OtherTok{=} \FloatTok{0.167}\NormalTok{, }\CommentTok{\# 1/2 of 1/3}
  \StringTok{\textasciigrave{}}\AttributeTok{person3}\StringTok{\textasciigrave{}} \OtherTok{=} \FloatTok{0.083}\NormalTok{, }\CommentTok{\# 1/4 of 1/3}
  \StringTok{\textasciigrave{}}\AttributeTok{person4}\StringTok{\textasciigrave{}} \OtherTok{=} \FloatTok{0.083}  \CommentTok{\# 1/4 of 1/3}
\NormalTok{  )}
\end{Highlighting}
\end{Shaded}

Assign household distribution by income and calculate :

\begin{Shaded}
\begin{Highlighting}[]
\CommentTok{\# Distribution of households by income}
\NormalTok{sB\_income }\OtherTok{\textless{}{-}} \FunctionTok{c}\NormalTok{(}\StringTok{\textasciigrave{}}\AttributeTok{ssi}\StringTok{\textasciigrave{}} \OtherTok{=} \FloatTok{0.5}\NormalTok{, }\StringTok{\textasciigrave{}}\AttributeTok{1500}\StringTok{\textasciigrave{}} \OtherTok{=} \FloatTok{0.5}\NormalTok{)}
\end{Highlighting}
\end{Shaded}

Because SSI amounts depend on whether the beneficiary lives alone or is
married, their affordable rents will vary. However, due to the
relatively small share of households in this model that could include
two married adults, we will assume that any persons enrolled in SSI
receive benefits for an individual. The current monthly SSI amount for
an eligible individual is \$943.\footnote{\href{https://www.ssa.gov/oact/cola/SSI.html}{SSI
  Federal Payment Amounts for 2024} (Accessed 2024-01-19)}

\begin{Shaded}
\begin{Highlighting}[]
\CommentTok{\# Monthly SSI income for eligible individual}
\NormalTok{sB\_ssi }\OtherTok{\textless{}{-}} \DecValTok{943}
\end{Highlighting}
\end{Shaded}

\newpage

\subsection{Calculations}\label{calculations-1}

Calculate affordable rents for SSI income and \$1,500/month
(\texttt{income\_source}) for households with 1 to 4 persons
(\texttt{hh\_size}) to determine monthly subsidy amounts
(\texttt{subsidy}):

\begin{Shaded}
\begin{Highlighting}[]
\CommentTok{\# Monthly subsidy about by household size}
\NormalTok{sB\_subsidy }\OtherTok{\textless{}{-}}\NormalTok{ hud\_ami }\SpecialCharTok{|\textgreater{}} 
  \FunctionTok{filter}\NormalTok{(}
\NormalTok{    AMI }\SpecialCharTok{\%in\%} \FunctionTok{c}\NormalTok{(}\StringTok{"60\% AMI"}\NormalTok{),        }\CommentTok{\# 60\% AMI only}
    \FunctionTok{str\_detect}\NormalTok{(hh\_size, }\StringTok{"[1234]"}\NormalTok{) }\CommentTok{\# 1{-}4 person households only}
\NormalTok{    ) }\SpecialCharTok{|\textgreater{}} 
  \FunctionTok{mutate}\NormalTok{(}
    \AttributeTok{aff\_rent\_60ami =}\NormalTok{ income}\SpecialCharTok{/}\DecValTok{12} \SpecialCharTok{*} \FloatTok{0.3} \CommentTok{\# 30\% of monthly income}
\NormalTok{  ) }\SpecialCharTok{|\textgreater{}} 
  \FunctionTok{select}\NormalTok{(}\DecValTok{2}\NormalTok{, }\DecValTok{4}\NormalTok{) }\SpecialCharTok{|\textgreater{}} 
  \FunctionTok{mutate}\NormalTok{(}
    \StringTok{\textasciigrave{}}\AttributeTok{ssi}\StringTok{\textasciigrave{}} \OtherTok{=}\NormalTok{ sB\_ssi }\SpecialCharTok{*} \FloatTok{0.3}\NormalTok{,  }\CommentTok{\# 30\% of SSI}
    \StringTok{\textasciigrave{}}\AttributeTok{1500}\StringTok{\textasciigrave{}} \OtherTok{=} \DecValTok{1500} \SpecialCharTok{*} \FloatTok{0.3} \CommentTok{\# 30\% of $1,500}
\NormalTok{  ) }\SpecialCharTok{|\textgreater{}} 
  \FunctionTok{pivot\_longer}\NormalTok{(}
    \DecValTok{3}\SpecialCharTok{:}\DecValTok{4}\NormalTok{,}
    \AttributeTok{names\_to =} \StringTok{"income\_source"}\NormalTok{,}
    \AttributeTok{values\_to =} \StringTok{"aff\_rent"}
\NormalTok{  ) }\SpecialCharTok{|\textgreater{}} 
  \FunctionTok{mutate}\NormalTok{(}
    \AttributeTok{subsidy =}\NormalTok{ aff\_rent\_60ami }\SpecialCharTok{{-}}\NormalTok{ aff\_rent }\CommentTok{\# Calculate subsidy}
\NormalTok{  ) }\SpecialCharTok{|\textgreater{}} 
  \FunctionTok{select}\NormalTok{(}\DecValTok{3}\NormalTok{, }\DecValTok{1}\NormalTok{, }\DecValTok{5}\NormalTok{)}
\end{Highlighting}
\end{Shaded}

\begin{tabu} to \linewidth {>{\raggedright}X>{\raggedright}X>{\centering}X}
\toprule
\textbf{income\_source} & \textbf{hh\_size} & \textbf{subsidy}\\
\midrule
 & person1 & 1299.60\\
\cmidrule{2-3}
 & person2 & 1533.60\\
\cmidrule{2-3}
 & person3 & 1752.60\\
\cmidrule{2-3}
\multirow[t]{-4}{*}{\raggedright\arraybackslash ssi} & person4 & 1973.10\\
\cmidrule{1-3}
 & person1 & 1132.50\\
\cmidrule{2-3}
 & person2 & 1366.50\\
\cmidrule{2-3}
 & person3 & 1585.50\\
\cmidrule{2-3}
\multirow[t]{-4}{*}{\raggedright\arraybackslash 1500} & person4 & 1806.00\\
\bottomrule
\end{tabu}

\newpage

Tabulate unique shares for both income source and household size:

\begin{Shaded}
\begin{Highlighting}[]
\NormalTok{sB\_dist }\OtherTok{\textless{}{-}} \FunctionTok{expand.grid}\NormalTok{(}
  \AttributeTok{income\_source =} \FunctionTok{names}\NormalTok{(sB\_income),}
  \AttributeTok{hh\_size =} \FunctionTok{names}\NormalTok{(sB\_person)}
\NormalTok{  ) }\SpecialCharTok{|\textgreater{}}
  \FunctionTok{mutate}\NormalTok{(}\AttributeTok{pct =}\NormalTok{ sB\_income[income\_source] }\SpecialCharTok{*}\NormalTok{ sB\_person[hh\_size])}
\end{Highlighting}
\end{Shaded}

\begin{tabu} to \linewidth {>{\raggedright}X>{\raggedright}X>{\centering}X}
\toprule
\textbf{income\_source} & \textbf{hh\_size} & \textbf{pct}\\
\midrule
 & person1 & 0.3335\\
\cmidrule{2-3}
 & person2 & 0.0835\\
\cmidrule{2-3}
 & person3 & 0.0415\\
\cmidrule{2-3}
\multirow[t]{-4}{*}{\raggedright\arraybackslash ssi} & person4 & 0.0415\\
\cmidrule{1-3}
 & person1 & 0.3335\\
\cmidrule{2-3}
 & person2 & 0.0835\\
\cmidrule{2-3}
 & person3 & 0.0415\\
\cmidrule{2-3}
\multirow[t]{-4}{*}{\raggedright\arraybackslash 1500} & person4 & 0.0415\\
\bottomrule
\end{tabu}

\hfill\break
\hfill\break
Multiply each household share by the total number of households served
(150) to determine the respective number served for each group
(\texttt{hh\_served}), rounded to the nearest whole number:

\begin{Shaded}
\begin{Highlighting}[]
\NormalTok{sB\_served }\OtherTok{\textless{}{-}}\NormalTok{ sB\_dist }\SpecialCharTok{|\textgreater{}} 
  \FunctionTok{mutate}\NormalTok{(}\AttributeTok{hh\_served =} \FunctionTok{round}\NormalTok{(pct }\SpecialCharTok{*}\NormalTok{ sB\_hh\_served)) }
\end{Highlighting}
\end{Shaded}

\newpage

\subsection{Model results}\label{model-results-1}

Join the monthly subsidy amounts(\texttt{subsidy}) by income source and
household size and calculate annual subsidy per household
(\texttt{subsidy\_annual}). Multiply that figure by the number of
households served to calculate the rental assistance required
(\texttt{budget\_rent}), then re-total to account for overhead costs
(\texttt{budget\_total}):

\begin{Shaded}
\begin{Highlighting}[]
\NormalTok{sB\_budget }\OtherTok{\textless{}{-}}\NormalTok{ sB\_served }\SpecialCharTok{|\textgreater{}} 
  \FunctionTok{left\_join}\NormalTok{(sB\_subsidy, }\FunctionTok{join\_by}\NormalTok{(income\_source, hh\_size)) }\SpecialCharTok{|\textgreater{}} 
  \FunctionTok{mutate}\NormalTok{(}
    \AttributeTok{subsidy\_annual =}\NormalTok{ subsidy }\SpecialCharTok{*} \DecValTok{12}\NormalTok{,}
    \AttributeTok{budget\_rent =}\NormalTok{ subsidy\_annual }\SpecialCharTok{*}\NormalTok{ hh\_served,}
    \AttributeTok{budget\_total =}\NormalTok{ budget\_rent}\SpecialCharTok{/}\NormalTok{(}\DecValTok{1} \SpecialCharTok{{-}}\NormalTok{ sB\_overhead)}
\NormalTok{    ) }\SpecialCharTok{|\textgreater{}} 
  \FunctionTok{select}\NormalTok{(}\DecValTok{1}\NormalTok{, }\DecValTok{2}\NormalTok{, }\DecValTok{4}\NormalTok{, }\DecValTok{7}\NormalTok{, }\DecValTok{8}\NormalTok{) }\SpecialCharTok{|\textgreater{}} 
  \FunctionTok{arrange}\NormalTok{(}\FunctionTok{desc}\NormalTok{(income\_source)) }\SpecialCharTok{|\textgreater{}} 
  \FunctionTok{adorn\_totals}\NormalTok{()}
\end{Highlighting}
\end{Shaded}

\begin{table}[H]

\caption{\label{tbl-sB}Scenario B - Estimated annual budget by income
source and household size}

\centering{

\begin{tabu} to \linewidth {>{\raggedright}X>{\raggedright}X>{\centering}X>{\centering}X>{\centering}X}
\toprule
\multicolumn{3}{c}{ } & \multicolumn{2}{c}{Estimated budget} \\
\cmidrule(l{3pt}r{3pt}){4-5}
\textbf{Income source} & \textbf{Household size} & \textbf{Households served} & \textbf{Rental assistance} & \textbf{Overhead included}\\
\midrule
 & 1 person & 50 & \$779,760.00 & \$974,700.00\\
\cmidrule{2-5}
 & 2 person & 13 & \$239,241.60 & \$299,052.00\\
\cmidrule{2-5}
 & 3 person & 6 & \$126,187.20 & \$157,734.00\\
\cmidrule{2-5}
\multirow[t]{-4}{*}{\raggedright\arraybackslash SSI} & 4 person & 6 & \$142,063.20 & \$177,579.00\\
\cmidrule{1-5}
 & 1 person & 50 & \$679,500.00 & \$849,375.00\\
\cmidrule{2-5}
 & 2 person & 13 & \$213,174.00 & \$266,467.50\\
\cmidrule{2-5}
 & 3 person & 6 & \$114,156.00 & \$142,695.00\\
\cmidrule{2-5}
\multirow[t]{-4}{*}{\raggedright\arraybackslash \$1,500/month} & 4 person & 6 & \$130,032.00 & \$162,540.00\\
\cmidrule{1-5}
\textbf{Total} & \textbf{-} & \textbf{150} & \textbf{\$2,424,114.00} & \textbf{\$3,030,142.50}\\
\bottomrule
\end{tabu}

}

\end{table}%

\hfill\break

\begin{tcolorbox}[enhanced jigsaw, breakable, toprule=.15mm, opacitybacktitle=0.6, colframe=quarto-callout-note-color-frame, colback=white, opacityback=0, coltitle=black, colbacktitle=quarto-callout-note-color!10!white, rightrule=.15mm, toptitle=1mm, titlerule=0mm, title=\textcolor{quarto-callout-note-color}{\faInfo}\hspace{0.5em}{Scenario B results}, bottomtitle=1mm, bottomrule=.15mm, leftrule=.75mm, arc=.35mm, left=2mm]

\emph{Average annual program cost per household: \$20,200.95}

\hfill\break
Under Scenario B, a total of 150 households experiencing housing
insecurity are served. Although two-thirds are individuals who have
lower housing costs than households with children, the average cost per
household is over twice that of Scenario A, due to the deep level of
subsidy provided. The estimated annual cost for the rental assistance
alone is \$2,424,114.00. Coupled with a higher administrative overhead
for expanded case management (20\%), the total projected funding
required is \$3,030,142.50.

\end{tcolorbox}

\newpage

\section{Scenario C -- Reduce Severe Cost Burden for Lower-Income
Working
Families}\label{scenario-c-reduce-severe-cost-burden-for-lower-income-working-families}

This scenario outlines a LRSP with a total annual allocation of
\$500,000. The primary goal of the program is to reduce housing cost
burden among households with incomes between 30\% and 50\% AMI.
Households must have one working adult and one or more dependent. The
model uses the following inputs to estimate the number of households
served.

\begin{longtable}[]{@{}
  >{\raggedright\arraybackslash}p{(\columnwidth - 2\tabcolsep) * \real{0.4022}}
  >{\raggedright\arraybackslash}p{(\columnwidth - 2\tabcolsep) * \real{0.5978}}@{}}
\toprule\noalign{}
\begin{minipage}[b]{\linewidth}\raggedright
\textbf{Variable}
\end{minipage} & \begin{minipage}[b]{\linewidth}\raggedright
\textbf{Input}
\end{minipage} \\
\midrule\noalign{}
\endhead
\bottomrule\noalign{}
\endlastfoot
\emph{Total program budget} & \$500,000\newline \\
\emph{Eligibility} & Household income between 30\% and 50\% AMI\newline
At least one working adult\newline At least one dependent\newline \\
\emph{Subsidy amount} & Difference between the SAFMR and\newline
40\% AMI of gross household income\newline \\
\begin{minipage}[t]{\linewidth}\raggedright
\emph{Distribution of household}\\
\emph{sizes among participants}\strut
\end{minipage} & 15\% - 2-person\newline
25\% - 3-person\newline
25\% - 4-person\newline
25\% - 5-person\newline
10\% - 6-person\newline \\
\begin{minipage}[t]{\linewidth}\raggedright
\emph{Distribution of incomes}\\
\emph{among participants}\strut
\end{minipage} & 25\% - 30\% AMI\newline
50\% - 40\% AMI\newline
25\% - 50\% AMI\newline \\
\emph{Administrative overhead} & 15\% of total program budget \\
\end{longtable}

Notes:

\begin{itemize}
\tightlist
\item
  SAFMR refers to the Small Area Fair Market Rent as adopted by
  Alexandria Redevelopment and Housing Authority (ARHA) for 2023.
\item
  The affordable monthly rent is 40\% of gross household income, not the
  standard 30\%.
\item
  The subsidy calculated for each household is respective to their
  household size.
\item
  Families will only occupy 1-bedroom, 2-bedroom, or 3-bedroom units.
\item
  The administrative overhead includes housing-specific case management.
\end{itemize}

\newpage

\subsection{Inputs}\label{inputs-2}

Assign budget (dollars) and overhead costs (percent) variables:

\begin{Shaded}
\begin{Highlighting}[]
\CommentTok{\# Budget allocation}
\NormalTok{sC\_budget }\OtherTok{\textless{}{-}} \DecValTok{500000}

\CommentTok{\# Overhead percentage}
\NormalTok{sC\_overhead }\OtherTok{\textless{}{-}} \FloatTok{0.15}
\end{Highlighting}
\end{Shaded}

Assign household distributions by AMI and household size (number of
persons):

\begin{Shaded}
\begin{Highlighting}[]
\CommentTok{\# Distribution of households by AMI}
\NormalTok{sC\_ami }\OtherTok{\textless{}{-}} \FunctionTok{c}\NormalTok{(}\StringTok{\textasciigrave{}}\AttributeTok{30\% AMI}\StringTok{\textasciigrave{}} \OtherTok{=} \FloatTok{0.25}\NormalTok{, }\StringTok{\textasciigrave{}}\AttributeTok{40\% AMI}\StringTok{\textasciigrave{}} \OtherTok{=} \FloatTok{0.50}\NormalTok{, }\StringTok{\textasciigrave{}}\AttributeTok{50\% AMI}\StringTok{\textasciigrave{}} \OtherTok{=} \FloatTok{0.25}\NormalTok{)}

\CommentTok{\# Distribution of households by household size}
\NormalTok{sC\_person }\OtherTok{\textless{}{-}} \FunctionTok{tibble}\NormalTok{(}
  \AttributeTok{hh\_size =} \FunctionTok{paste0}\NormalTok{(}\StringTok{"person"}\NormalTok{, }\DecValTok{2}\SpecialCharTok{:}\DecValTok{6}\NormalTok{),}
  \AttributeTok{pct =} \FunctionTok{c}\NormalTok{(}\FloatTok{0.15}\NormalTok{, }\FloatTok{0.25}\NormalTok{, }\FloatTok{0.25}\NormalTok{, }\FloatTok{0.25}\NormalTok{, }\FloatTok{0.10}\NormalTok{)}
\NormalTok{)}
\end{Highlighting}
\end{Shaded}

Because subsidy amounts will be calculated using SAFMR, we also need to
estimate household shares across units by size (number of bedrooms). The
model will use the following assumptions:

\begin{itemize}
\tightlist
\item
  2-person: 100\% 1-bedroom
\item
  3-person: 10\% 1-bedroom, 90\% 2-bedroom
\item
  4-person: 80\% 2-bedroom, 20\% 3-bedroom
\item
  5-person: 100\% 3-bedroom
\item
  6-person: 100\% 3-bedroom
\end{itemize}

\begin{Shaded}
\begin{Highlighting}[]
\CommentTok{\# Distribution of household sizes by unit size}
\NormalTok{sC\_unit }\OtherTok{\textless{}{-}} \FunctionTok{list}\NormalTok{(}
    \StringTok{\textasciigrave{}}\AttributeTok{person2}\StringTok{\textasciigrave{}} \OtherTok{=} \FunctionTok{c}\NormalTok{(}\StringTok{\textasciigrave{}}\AttributeTok{bedroom1}\StringTok{\textasciigrave{}} \OtherTok{=} \DecValTok{1}\NormalTok{),}
    \StringTok{\textasciigrave{}}\AttributeTok{person3}\StringTok{\textasciigrave{}} \OtherTok{=} \FunctionTok{c}\NormalTok{(}\StringTok{\textasciigrave{}}\AttributeTok{bedroom1}\StringTok{\textasciigrave{}} \OtherTok{=} \FloatTok{0.1}\NormalTok{, }\StringTok{\textasciigrave{}}\AttributeTok{bedroom2}\StringTok{\textasciigrave{}} \OtherTok{=} \FloatTok{0.9}\NormalTok{),}
    \StringTok{\textasciigrave{}}\AttributeTok{person4}\StringTok{\textasciigrave{}} \OtherTok{=} \FunctionTok{c}\NormalTok{(}\StringTok{\textasciigrave{}}\AttributeTok{bedroom2}\StringTok{\textasciigrave{}} \OtherTok{=} \FloatTok{0.8}\NormalTok{, }\StringTok{\textasciigrave{}}\AttributeTok{bedroom3}\StringTok{\textasciigrave{}} \OtherTok{=} \FloatTok{0.2}\NormalTok{),}
    \StringTok{\textasciigrave{}}\AttributeTok{person5}\StringTok{\textasciigrave{}} \OtherTok{=} \FunctionTok{c}\NormalTok{(}\StringTok{\textasciigrave{}}\AttributeTok{bedroom3}\StringTok{\textasciigrave{}} \OtherTok{=} \DecValTok{1}\NormalTok{),}
    \StringTok{\textasciigrave{}}\AttributeTok{person6}\StringTok{\textasciigrave{}} \OtherTok{=} \FunctionTok{c}\NormalTok{(}\StringTok{\textasciigrave{}}\AttributeTok{bedroom3}\StringTok{\textasciigrave{}} \OtherTok{=} \DecValTok{1}\NormalTok{)}
\NormalTok{  )}
\end{Highlighting}
\end{Shaded}

\begin{tabu} to \linewidth {>{\raggedright}X>{\centering}X>{\centering}X>{\centering}X}
\toprule
\textbf{persons} & \textbf{bedroom1} & \textbf{bedroom2} & \textbf{bedroom3}\\
\midrule
person2 & 1 & - & -\\
person3 & 0.1 & 0.9 & -\\
person4 & - & 0.8 & 0.2\\
person5 & - & - & 1\\
person6 & - & - & 1\\
\bottomrule
\end{tabu}

\newpage

Create data frame with all permutations for income, unit size, and
household sizes. Exclude non-valid combinations of unit and household
size:

\begin{Shaded}
\begin{Highlighting}[]
\NormalTok{sC\_hh\_type }\OtherTok{\textless{}{-}} \FunctionTok{expand.grid}\NormalTok{(}
  \AttributeTok{AMI =} \FunctionTok{names}\NormalTok{(sC\_ami),}
  \AttributeTok{bedrooms =} \FunctionTok{unlist}\NormalTok{(}\FunctionTok{lapply}\NormalTok{(}\FunctionTok{names}\NormalTok{(sC\_unit), }\ControlFlowTok{function}\NormalTok{(unit) }\FunctionTok{names}\NormalTok{(sC\_unit[[unit]]))),}
  \AttributeTok{hh\_size =}\NormalTok{ sC\_person}\SpecialCharTok{$}\NormalTok{hh\_size}
\NormalTok{  ) }\SpecialCharTok{|\textgreater{}}
  \FunctionTok{distinct}\NormalTok{() }\SpecialCharTok{|\textgreater{}} 
  \FunctionTok{filter}\NormalTok{(}
    \SpecialCharTok{!}\NormalTok{(bedrooms }\SpecialCharTok{==} \StringTok{"bedroom1"} \SpecialCharTok{\&} \SpecialCharTok{!}\NormalTok{hh\_size }\SpecialCharTok{\%in\%} \FunctionTok{c}\NormalTok{(}\StringTok{"person2"}\NormalTok{, }\StringTok{"person3"}\NormalTok{)),}
    \SpecialCharTok{!}\NormalTok{(bedrooms }\SpecialCharTok{==} \StringTok{"bedroom2"} \SpecialCharTok{\&}\NormalTok{ hh\_size }\SpecialCharTok{\%in\%} \FunctionTok{c}\NormalTok{(}\StringTok{"person2"}\NormalTok{, }\StringTok{"person5"}\NormalTok{, }\StringTok{"person6"}\NormalTok{)),}
    \SpecialCharTok{!}\NormalTok{(bedrooms }\SpecialCharTok{==} \StringTok{"bedroom3"} \SpecialCharTok{\&}\NormalTok{ hh\_size }\SpecialCharTok{\%in\%} \FunctionTok{c}\NormalTok{(}\StringTok{"person2"}\NormalTok{, }\StringTok{"person3"}\NormalTok{))}
\NormalTok{  )}
\end{Highlighting}
\end{Shaded}

\subsection{Calculations}\label{calculations-2}

Create a function to tabulates respective household distributions by
AMI, household size, and unit type:

\begin{Shaded}
\begin{Highlighting}[]
\NormalTok{sC\_dist\_fn }\OtherTok{\textless{}{-}} \ControlFlowTok{function}\NormalTok{() \{}
  
  \CommentTok{\# Build data frame with AMI and household size distributions}
\NormalTok{  dist }\OtherTok{\textless{}{-}} \FunctionTok{expand.grid}\NormalTok{(}
    \AttributeTok{AMI =} \FunctionTok{names}\NormalTok{(sC\_ami),}
    \AttributeTok{hh\_size =}\NormalTok{ sC\_person}\SpecialCharTok{$}\NormalTok{hh\_size}
\NormalTok{  ) }\SpecialCharTok{|\textgreater{}}
    \CommentTok{\# Match hh\_size with sC\_person$hh\_size to fetch the correct percentage}
    \FunctionTok{mutate}\NormalTok{(}\AttributeTok{households =}\NormalTok{ sC\_ami[AMI] }\SpecialCharTok{*}\NormalTok{ sC\_person}\SpecialCharTok{$}\NormalTok{pct[}\FunctionTok{match}\NormalTok{(hh\_size, sC\_person}\SpecialCharTok{$}\NormalTok{hh\_size)])}

  \CommentTok{\# Initialize an empty data frame for the final distribution}
\NormalTok{  final\_distribution }\OtherTok{\textless{}{-}} \FunctionTok{data.frame}\NormalTok{()}

  \CommentTok{\# Iterate distribution by unit size}
  \ControlFlowTok{for}\NormalTok{ (person }\ControlFlowTok{in} \FunctionTok{unique}\NormalTok{(sC\_hh\_type}\SpecialCharTok{$}\NormalTok{hh\_size)) \{}
\NormalTok{    current\_dist }\OtherTok{\textless{}{-}} \FunctionTok{subset}\NormalTok{(dist, hh\_size }\SpecialCharTok{==}\NormalTok{ person)}

    \ControlFlowTok{for}\NormalTok{ (bedroom\_count }\ControlFlowTok{in} \FunctionTok{names}\NormalTok{(sC\_unit[[person]])) \{}
\NormalTok{      current\_dist}\SpecialCharTok{$}\NormalTok{bedrooms }\OtherTok{\textless{}{-}} \FunctionTok{as.character}\NormalTok{(bedroom\_count)}

\NormalTok{      current\_dist}\SpecialCharTok{$}\NormalTok{pct }\OtherTok{\textless{}{-}}\NormalTok{ current\_dist}\SpecialCharTok{$}\NormalTok{households }\SpecialCharTok{*}
\NormalTok{        sC\_unit[[person]][bedroom\_count]}

\NormalTok{      final\_distribution }\OtherTok{\textless{}{-}} \FunctionTok{rbind}\NormalTok{(}
\NormalTok{        final\_distribution,}
\NormalTok{        current\_dist[, }\FunctionTok{c}\NormalTok{(}\StringTok{"AMI"}\NormalTok{, }\StringTok{"bedrooms"}\NormalTok{, }\StringTok{"hh\_size"}\NormalTok{, }\StringTok{"pct"}\NormalTok{)]}
\NormalTok{      )}
\NormalTok{    \}}
\NormalTok{  \}}

  \FunctionTok{return}\NormalTok{(final\_distribution)}
\NormalTok{\}}

\NormalTok{sC\_dist }\OtherTok{\textless{}{-}} \FunctionTok{sC\_dist\_fn}\NormalTok{()}
\end{Highlighting}
\end{Shaded}

\newpage

Calculate affordable rents (\texttt{aff\_rent}) at 30\% AMI, 40\% AMI,
and 50\% AMI for households with 2 to 6 persons (\texttt{hh\_size}):

\begin{Shaded}
\begin{Highlighting}[]
\CommentTok{\# Monthly affordable rents by household size}
\NormalTok{sC\_aff\_rents }\OtherTok{\textless{}{-}}\NormalTok{ hud\_ami }\SpecialCharTok{|\textgreater{}} 
  \FunctionTok{filter}\NormalTok{(}
\NormalTok{    AMI }\SpecialCharTok{\%in\%} \FunctionTok{c}\NormalTok{(}\StringTok{"30\% AMI"}\NormalTok{, }\StringTok{"40\% AMI"}\NormalTok{, }\StringTok{"50\% AMI"}\NormalTok{), }\CommentTok{\# 30\%, 40\%, and 50\% AMI only}
    \FunctionTok{str\_detect}\NormalTok{(hh\_size, }\StringTok{"[23456]"}\NormalTok{) }\CommentTok{\# 2{-}6 person households only}
\NormalTok{    ) }\SpecialCharTok{|\textgreater{}} 
  \FunctionTok{mutate}\NormalTok{(}
    \AttributeTok{aff\_rent =}\NormalTok{ income}\SpecialCharTok{/}\DecValTok{12} \SpecialCharTok{*} \FloatTok{0.4} \CommentTok{\# 40\% of monthly income}
\NormalTok{  ) }\SpecialCharTok{|\textgreater{}} 
  \FunctionTok{select}\NormalTok{(}\SpecialCharTok{{-}}\DecValTok{3}\NormalTok{)}
\end{Highlighting}
\end{Shaded}

\begingroup\fontsize{8}{10}\selectfont

\begin{tabu} to \linewidth {>{\raggedright}X>{\raggedright}X>{\centering}X}
\toprule
\textbf{AMI} & \textbf{hh\_size} & \textbf{aff\_rent}\\
\midrule
 & person2 & 1206.00\\
\cmidrule{2-3}
 & person3 & 1357.00\\
\cmidrule{2-3}
 & person4 & 1507.00\\
\cmidrule{2-3}
 & person5 & 1628.00\\
\cmidrule{2-3}
\multirow[t]{-5}{*}{\raggedright\arraybackslash 30\% AMI} & person6 & 1749.00\\
\cmidrule{1-3}
 & person2 & 1608.00\\
\cmidrule{2-3}
 & person3 & 1809.33\\
\cmidrule{2-3}
 & person4 & 2009.33\\
\cmidrule{2-3}
 & person5 & 2173.67\\
\cmidrule{2-3}
\multirow[t]{-5}{*}{\raggedright\arraybackslash 40\% AMI} & person6 & 2332.00\\
\cmidrule{1-3}
 & person2 & 2010.00\\
\cmidrule{2-3}
 & person3 & 2261.67\\
\cmidrule{2-3}
 & person4 & 2511.67\\
\cmidrule{2-3}
 & person5 & 2713.33\\
\cmidrule{2-3}
\multirow[t]{-5}{*}{\raggedright\arraybackslash 50\% AMI} & person6 & 2915.00\\
\bottomrule
\end{tabu}
\endgroup{}

\newpage

Join \texttt{sC\_hh\_type} with average FMR by unit size
(\texttt{fmrs\_avg}) and affordable rents (\texttt{sC\_aff\_rents}),
then find difference between values to calculate the monthly subsidy
(\texttt{subsidy}):

\begin{Shaded}
\begin{Highlighting}[]
\NormalTok{sC\_subsidy }\OtherTok{\textless{}{-}}\NormalTok{ sC\_hh\_type }\SpecialCharTok{|\textgreater{}} 
  \FunctionTok{left\_join}\NormalTok{(sC\_aff\_rents, }\FunctionTok{join\_by}\NormalTok{(AMI, hh\_size)) }\SpecialCharTok{|\textgreater{}} 
  \FunctionTok{left\_join}\NormalTok{(fmrs\_avg) }\SpecialCharTok{|\textgreater{}} 
  \FunctionTok{mutate}\NormalTok{(}\AttributeTok{subsidy =}\NormalTok{ fmr\_avg }\SpecialCharTok{{-}}\NormalTok{ aff\_rent)}
\end{Highlighting}
\end{Shaded}

\begingroup\fontsize{8}{10}\selectfont

\begin{tabu} to \linewidth {>{\raggedright}X>{\raggedright}X>{\raggedright}X>{\centering}X>{\centering}X>{}c}
\toprule
\textbf{AMI} & \textbf{bedrooms} & \textbf{hh\_size} & \textbf{aff\_rent} & \textbf{fmr\_avg} & \textbf{subsidy}\\
\midrule
 &  & person2 & 1206.00 & 2019.88 & \textcolor{black}{813.88}\\
\cmidrule{3-6}
 & \multirow[t]{-2}{*}{\raggedright\arraybackslash bedroom1} & person3 & 1357.00 & 2019.88 & \textcolor{black}{662.88}\\
\cmidrule{2-6}
 &  & person3 & 1357.00 & 2300.38 & \textcolor{black}{943.38}\\
\cmidrule{3-6}
 & \multirow[t]{-2}{*}{\raggedright\arraybackslash bedroom2} & person4 & 1507.00 & 2300.38 & \textcolor{black}{793.38}\\
\cmidrule{2-6}
 &  & person4 & 1507.00 & 2876.50 & \textcolor{black}{1369.50}\\
\cmidrule{3-6}
 &  & person5 & 1628.00 & 2876.50 & \textcolor{black}{1248.50}\\
\cmidrule{3-6}
\multirow[t]{-7}{*}[2\dimexpr\aboverulesep+\belowrulesep+\cmidrulewidth]{\raggedright\arraybackslash 30\% AMI} & \multirow[t]{-3}{*}{\raggedright\arraybackslash bedroom3} & person6 & 1749.00 & 2876.50 & \textcolor{black}{1127.50}\\
\cmidrule{1-6}
 &  & person2 & 1608.00 & 2019.88 & \textcolor{black}{411.88}\\
\cmidrule{3-6}
 & \multirow[t]{-2}{*}{\raggedright\arraybackslash bedroom1} & person3 & 1809.33 & 2019.88 & \textcolor{black}{210.54}\\
\cmidrule{2-6}
 &  & person3 & 1809.33 & 2300.38 & \textcolor{black}{491.04}\\
\cmidrule{3-6}
 & \multirow[t]{-2}{*}{\raggedright\arraybackslash bedroom2} & person4 & 2009.33 & 2300.38 & \textcolor{black}{291.04}\\
\cmidrule{2-6}
 &  & person4 & 2009.33 & 2876.50 & \textcolor{black}{867.17}\\
\cmidrule{3-6}
 &  & person5 & 2173.67 & 2876.50 & \textcolor{black}{702.83}\\
\cmidrule{3-6}
\multirow[t]{-7}{*}[2\dimexpr\aboverulesep+\belowrulesep+\cmidrulewidth]{\raggedright\arraybackslash 40\% AMI} & \multirow[t]{-3}{*}{\raggedright\arraybackslash bedroom3} & person6 & 2332.00 & 2876.50 & \textcolor{black}{544.50}\\
\cmidrule{1-6}
 &  & person2 & 2010.00 & 2019.88 & \textcolor{red}{9.88}\\
\cmidrule{3-6}
 & \multirow[t]{-2}{*}{\raggedright\arraybackslash bedroom1} & person3 & 2261.67 & 2019.88 & \textcolor{red}{-241.79}\\
\cmidrule{2-6}
 &  & person3 & 2261.67 & 2300.38 & \textcolor{red}{38.71}\\
\cmidrule{3-6}
 & \multirow[t]{-2}{*}{\raggedright\arraybackslash bedroom2} & person4 & 2511.67 & 2300.38 & \textcolor{red}{-211.29}\\
\cmidrule{2-6}
 &  & person4 & 2511.67 & 2876.50 & \textcolor{black}{364.83}\\
\cmidrule{3-6}
 &  & person5 & 2713.33 & 2876.50 & \textcolor{black}{163.17}\\
\cmidrule{3-6}
\multirow[t]{-7}{*}[2\dimexpr\aboverulesep+\belowrulesep+\cmidrulewidth]{\raggedright\arraybackslash 50\% AMI} & \multirow[t]{-3}{*}{\raggedright\arraybackslash bedroom3} & person6 & 2915.00 & 2876.50 & \textcolor{red}{-38.50}\\
\bottomrule
\end{tabu}
\endgroup{}

\hfill\break

\begin{tcolorbox}[enhanced jigsaw, breakable, toprule=.15mm, opacitybacktitle=0.6, colframe=quarto-callout-warning-color-frame, colback=white, opacityback=0, coltitle=black, colbacktitle=quarto-callout-warning-color!10!white, rightrule=.15mm, toptitle=1mm, titlerule=0mm, title=\textcolor{quarto-callout-warning-color}{\faExclamationTriangle}\hspace{0.5em}{Some affordable rents almost equal to FMRs}, bottomtitle=1mm, bottomrule=.15mm, leftrule=.75mm, arc=.35mm, left=2mm]

Note that the subsidies for 5 household types are negligible --- under
\$50. (See red values.) These cases are the result of higher affordable
rents among those earning 50\% AMI or more, along with the higher 40\%
tenant contribution.

\end{tcolorbox}

\hfill\break
For the purposes of this model, these household types with no or very
little subsidy need will be excluded. The sum of their respective shares
will be redistributed to the remaining 16 household combinations.

\newpage

Calculate total shares of excluded and remaining household types:

\begin{Shaded}
\begin{Highlighting}[]
\NormalTok{sC\_excl }\OtherTok{\textless{}{-}}\NormalTok{ sC\_subsidy }\SpecialCharTok{|\textgreater{}} 
  \FunctionTok{left\_join}\NormalTok{(sC\_dist, }\FunctionTok{join\_by}\NormalTok{(AMI, bedrooms, hh\_size)) }\SpecialCharTok{|\textgreater{}} 
  \FunctionTok{mutate}\NormalTok{(}
    \AttributeTok{status =} \FunctionTok{case\_when}\NormalTok{(}
\NormalTok{      subsidy }\SpecialCharTok{\textless{}} \DecValTok{50} \SpecialCharTok{\textasciitilde{}} \StringTok{"exclude"}\NormalTok{,}
\NormalTok{      subsidy }\SpecialCharTok{\textgreater{}} \DecValTok{50} \SpecialCharTok{\textasciitilde{}} \StringTok{"retain"}
\NormalTok{    ),}
    \AttributeTok{.before =} \DecValTok{7}
\NormalTok{  )}
\end{Highlighting}
\end{Shaded}

\begin{tabu} to \linewidth {>{\raggedright}X>{\centering}X}
\toprule
\textbf{status} & \textbf{pct}\\
\midrule
retain & 0.825\\
\textcolor{red}{exclude} & \textcolor{red}{0.175}\\
\bottomrule
\end{tabu}

\hfill\break
\hfill\break
Evenly distributing this 0.175 across the remaining 16 combinations
would not respect the original group distributions by AMI and household
size. Therefore, this surplus share will be manually redistributed to
each remaining combination to ensure the new AMI and household size
group subtotals are as close to the original values as possible.

First, we determine the change in shares by each AMI and household size
group resulting from excluding the 5 invalid combinations.

\begin{Shaded}
\begin{Highlighting}[]
\NormalTok{sC\_excl\_grp }\OtherTok{\textless{}{-}}\NormalTok{ sC\_excl }\SpecialCharTok{|\textgreater{}} 
  \FunctionTok{select}\NormalTok{(}\DecValTok{1}\SpecialCharTok{:}\DecValTok{3}\NormalTok{, }\DecValTok{7}\SpecialCharTok{:}\DecValTok{8}\NormalTok{) }\SpecialCharTok{|\textgreater{}} 
  \FunctionTok{mutate}\NormalTok{(}
    \AttributeTok{pct\_excl =}
      \FunctionTok{case\_when}\NormalTok{(}
\NormalTok{        status }\SpecialCharTok{==} \StringTok{"retain"} \SpecialCharTok{\textasciitilde{}}\NormalTok{ pct,}
\NormalTok{        status }\SpecialCharTok{==} \StringTok{"exclude"} \SpecialCharTok{\textasciitilde{}} \DecValTok{0}
\NormalTok{      )}
\NormalTok{  )}
\end{Highlighting}
\end{Shaded}

Change in shares grouped by AMI:

\begin{tabu} to \linewidth {>{\raggedright}X>{\centering}X>{\centering}X>{}c}
\toprule
\textbf{AMI} & \textbf{pct} & \textbf{pct\_excl} & \textbf{diff}\\
\midrule
30\% AMI & 0.25 & 0.250 & \textcolor{black}{0.000}\\
40\% AMI & 0.50 & 0.500 & \textcolor{black}{0.000}\\
50\% AMI & 0.25 & 0.075 & \textcolor{red}{-0.175}\\
\bottomrule
\end{tabu}

\hfill\break
\hfill\break
Change in shares grouped by household size:

\begin{tabu} to \linewidth {>{\raggedright}X>{\centering}X>{\centering}X>{}c}
\toprule
\textbf{hh\_size} & \textbf{pct} & \textbf{pct\_excl} & \textbf{diff}\\
\midrule
person2 & 0.15 & 0.1125 & \textcolor{red}{-0.0375}\\
person3 & 0.25 & 0.1875 & \textcolor{red}{-0.0625}\\
person4 & 0.25 & 0.2000 & \textcolor{red}{-0.0500}\\
person5 & 0.25 & 0.2500 & \textcolor{black}{0.0000}\\
person6 & 0.10 & 0.0750 & \textcolor{red}{-0.0250}\\
\bottomrule
\end{tabu}

\newpage

In redistributing these shares, we will need to:

\begin{itemize}
\tightlist
\item
  Significantly increase the remaining 50\% AMI household types
  (4-person and 5-person in 3-bedroom units) to maintain overall balance
  across AMI groups
\item
  Reduce the corresponding shares in both 30\% AMI and 40\% AMI groups
  to maintain overall balance across household sizes
\item
  Slightly increase the 2-person, 3-person, and 6-person household types
  in both 30\% AMI and 40\% AMI groups to account for losses within 50\%
  AMI category
\end{itemize}

To accomplish this, the excluded share is divided into 16 parts (\(p\)):

\[p = \frac{0.175}{16} = 0.0109375\]

We can increase or decrease each household combination by a multiple of
\(p\) as long as the net increase across all households is \(+16p\).

The following allocation reproduces the original AMI distribution, and a
new household size distribution where each category is within
\(\pm 0.02\) of the original share.

\begin{Shaded}
\begin{Highlighting}[]
\NormalTok{sC\_redist }\OtherTok{\textless{}{-}}\NormalTok{ sC\_excl\_grp }\SpecialCharTok{|\textgreater{}} 
  \FunctionTok{filter}\NormalTok{(status }\SpecialCharTok{==} \StringTok{"retain"}\NormalTok{) }\SpecialCharTok{|\textgreater{}} 
  \FunctionTok{select}\NormalTok{(}\DecValTok{1}\NormalTok{, }\DecValTok{2}\NormalTok{, }\DecValTok{3}\NormalTok{, }\DecValTok{5}\NormalTok{) }\SpecialCharTok{|\textgreater{}} 
  \FunctionTok{arrange}\NormalTok{(AMI, hh\_size, bedrooms) }\SpecialCharTok{|\textgreater{}} 
  \FunctionTok{mutate}\NormalTok{(}
    \AttributeTok{p\_shares =} \FunctionTok{case\_when}\NormalTok{(}
\NormalTok{      AMI }\SpecialCharTok{==} \StringTok{"30\% AMI"} \SpecialCharTok{\&}\NormalTok{ bedrooms }\SpecialCharTok{==} \StringTok{"bedroom2"} \SpecialCharTok{\&}\NormalTok{ hh\_size }\SpecialCharTok{==} \StringTok{"person4"} \SpecialCharTok{\textasciitilde{}} \DecValTok{0}\NormalTok{,}
\NormalTok{      AMI }\SpecialCharTok{==} \StringTok{"30\% AMI"} \SpecialCharTok{\&}\NormalTok{ bedrooms }\SpecialCharTok{==} \StringTok{"bedroom3"} \SpecialCharTok{\&}\NormalTok{ hh\_size }\SpecialCharTok{==} \StringTok{"person4"} \SpecialCharTok{\textasciitilde{}} \SpecialCharTok{{-}}\DecValTok{1}\NormalTok{,}
\NormalTok{      AMI }\SpecialCharTok{==} \StringTok{"30\% AMI"} \SpecialCharTok{\&}\NormalTok{ bedrooms }\SpecialCharTok{==} \StringTok{"bedroom3"} \SpecialCharTok{\&}\NormalTok{ hh\_size }\SpecialCharTok{==} \StringTok{"person5"} \SpecialCharTok{\textasciitilde{}} \SpecialCharTok{{-}}\DecValTok{3}\NormalTok{,}
\NormalTok{      AMI }\SpecialCharTok{==} \StringTok{"40\% AMI"} \SpecialCharTok{\&}\NormalTok{ bedrooms }\SpecialCharTok{==} \StringTok{"bedroom3"} \SpecialCharTok{\&}\NormalTok{ hh\_size }\SpecialCharTok{==} \StringTok{"person4"} \SpecialCharTok{\textasciitilde{}} \SpecialCharTok{{-}}\DecValTok{2}\NormalTok{,}
\NormalTok{      AMI }\SpecialCharTok{==} \StringTok{"40\% AMI"} \SpecialCharTok{\&}\NormalTok{ bedrooms }\SpecialCharTok{==} \StringTok{"bedroom3"} \SpecialCharTok{\&}\NormalTok{ hh\_size }\SpecialCharTok{==} \StringTok{"person5"} \SpecialCharTok{\textasciitilde{}} \SpecialCharTok{{-}}\DecValTok{4}\NormalTok{,}
\NormalTok{      AMI }\SpecialCharTok{==} \StringTok{"40\% AMI"} \SpecialCharTok{\&}\NormalTok{ bedrooms }\SpecialCharTok{==} \StringTok{"bedroom3"} \SpecialCharTok{\&}\NormalTok{ hh\_size }\SpecialCharTok{==} \StringTok{"person6"} \SpecialCharTok{\textasciitilde{}} \DecValTok{2}\NormalTok{,}
\NormalTok{      AMI }\SpecialCharTok{==} \StringTok{"50\% AMI"} \SpecialCharTok{\textasciitilde{}} \DecValTok{8}\NormalTok{,}
      \AttributeTok{.default =} \DecValTok{1}
\NormalTok{    ),}
    \AttributeTok{pct\_redist =}\NormalTok{ pct }\SpecialCharTok{+}\NormalTok{ (p}\SpecialCharTok{*}\NormalTok{p\_shares)}
\NormalTok{  )}
\end{Highlighting}
\end{Shaded}

\begingroup\fontsize{8}{10}\selectfont

\begin{tabu} to \linewidth {>{\raggedright}X>{\raggedright}X>{\raggedright}X>{\centering}X>{\centering}X>{\centering}X}
\toprule
\textbf{AMI} & \textbf{bedrooms} & \textbf{hh\_size} & \textbf{pct} & \textbf{p\_shares} & \textbf{pct\_redist}\\
\midrule
 &  & person2 & 0.0375 & 1 & 0.0484\\
\cmidrule{3-6}
 & \multirow[t]{-2}{*}{\raggedright\arraybackslash bedroom1} & person3 & 0.0063 & 1 & 0.0172\\
\cmidrule{2-6}
 &  & person3 & 0.0562 & 1 & 0.0672\\
\cmidrule{3-6}
 & \multirow[t]{-2}{*}{\raggedright\arraybackslash bedroom2} & person4 & 0.0500 & 0 & 0.0500\\
\cmidrule{2-6}
 &  & person4 & 0.0125 & -1 & 0.0016\\
\cmidrule{3-6}
 &  & person5 & 0.0625 & -3 & 0.0297\\
\cmidrule{3-6}
\multirow[t]{-7}{*}[2\dimexpr\aboverulesep+\belowrulesep+\cmidrulewidth]{\raggedright\arraybackslash 30\% AMI} & \multirow[t]{-3}{*}{\raggedright\arraybackslash bedroom3} & person6 & 0.0250 & 1 & 0.0359\\
\cmidrule{1-6}
 &  & person2 & 0.0750 & 1 & 0.0859\\
\cmidrule{3-6}
 & \multirow[t]{-2}{*}{\raggedright\arraybackslash bedroom1} & person3 & 0.0125 & 1 & 0.0234\\
\cmidrule{2-6}
 &  & person3 & 0.1125 & 1 & 0.1234\\
\cmidrule{3-6}
 & \multirow[t]{-2}{*}{\raggedright\arraybackslash bedroom2} & person4 & 0.1000 & 1 & 0.1109\\
\cmidrule{2-6}
 &  & person4 & 0.0250 & -2 & 0.0031\\
\cmidrule{3-6}
 &  & person5 & 0.1250 & -4 & 0.0813\\
\cmidrule{3-6}
\multirow[t]{-7}{*}[2\dimexpr\aboverulesep+\belowrulesep+\cmidrulewidth]{\raggedright\arraybackslash 40\% AMI} & \multirow[t]{-3}{*}{\raggedright\arraybackslash bedroom3} & person6 & 0.0500 & 2 & 0.0719\\
\cmidrule{1-6}
 &  & person4 & 0.0125 & 8 & 0.1000\\
\cmidrule{3-6}
\multirow[t]{-2}{*}{\raggedright\arraybackslash 50\% AMI} & \multirow[t]{-2}{*}{\raggedright\arraybackslash bedroom3} & person5 & 0.0625 & 8 & 0.1500\\
\bottomrule
\end{tabu}
\endgroup{}

\newpage

Original distribution of households by AMI retained:

\begin{tabu} to \linewidth {>{\raggedright}X>{\centering}X}
\toprule
\textbf{AMI} & \textbf{pct\_redist}\\
\midrule
30\% AMI & 0.25\\
40\% AMI & 0.50\\
50\% AMI & 0.25\\
\bottomrule
\end{tabu}

\hfill\break
\hfill\break
New distribution of households by household size:

\begin{tabu} to \linewidth {>{\raggedright}X>{\centering}X}
\toprule
\textbf{hh\_size} & \textbf{pct\_redist}\\
\midrule
person2 & 0.1344\\
person3 & 0.2313\\
person4 & 0.2656\\
person5 & 0.2609\\
person6 & 0.1078\\
\bottomrule
\end{tabu}

\hfill\break
\hfill\break
Rejoin the redistributed households with the calculated subsidy amounts
per household:

\begin{Shaded}
\begin{Highlighting}[]
\NormalTok{sC\_redist\_subsidy }\OtherTok{\textless{}{-}}\NormalTok{ sC\_subsidy }\SpecialCharTok{|\textgreater{}} 
  \FunctionTok{right\_join}\NormalTok{(sC\_redist, }\FunctionTok{join\_by}\NormalTok{(AMI, bedrooms, hh\_size)) }\SpecialCharTok{|\textgreater{}} 
  \FunctionTok{select}\NormalTok{(}\DecValTok{1}\SpecialCharTok{:}\DecValTok{3}\NormalTok{, }\DecValTok{6}\NormalTok{, }\DecValTok{9}\NormalTok{)}
\end{Highlighting}
\end{Shaded}

\begingroup\fontsize{8}{10}\selectfont

\begin{tabu} to \linewidth {>{\raggedright}X>{\raggedright}X>{\raggedright}X>{\centering}X>{\centering}X}
\toprule
\textbf{AMI} & \textbf{bedrooms} & \textbf{hh\_size} & \textbf{subsidy} & \textbf{pct\_redist}\\
\midrule
 &  & person2 & 813.88 & 0.0484\\
\cmidrule{3-5}
 & \multirow[t]{-2}{*}{\raggedright\arraybackslash bedroom1} & person3 & 662.88 & 0.0172\\
\cmidrule{2-5}
 &  & person3 & 943.38 & 0.0672\\
\cmidrule{3-5}
 & \multirow[t]{-2}{*}{\raggedright\arraybackslash bedroom2} & person4 & 793.38 & 0.0500\\
\cmidrule{2-5}
 &  & person4 & 1369.50 & 0.0016\\
\cmidrule{3-5}
 &  & person5 & 1248.50 & 0.0297\\
\cmidrule{3-5}
\multirow[t]{-7}{*}[2\dimexpr\aboverulesep+\belowrulesep+\cmidrulewidth]{\raggedright\arraybackslash 30\% AMI} & \multirow[t]{-3}{*}{\raggedright\arraybackslash bedroom3} & person6 & 1127.50 & 0.0359\\
\cmidrule{1-5}
 &  & person2 & 411.88 & 0.0859\\
\cmidrule{3-5}
 & \multirow[t]{-2}{*}{\raggedright\arraybackslash bedroom1} & person3 & 210.54 & 0.0234\\
\cmidrule{2-5}
 &  & person3 & 491.04 & 0.1234\\
\cmidrule{3-5}
 & \multirow[t]{-2}{*}{\raggedright\arraybackslash bedroom2} & person4 & 291.04 & 0.1109\\
\cmidrule{2-5}
 &  & person4 & 867.17 & 0.0031\\
\cmidrule{3-5}
 &  & person5 & 702.83 & 0.0813\\
\cmidrule{3-5}
\multirow[t]{-7}{*}[2\dimexpr\aboverulesep+\belowrulesep+\cmidrulewidth]{\raggedright\arraybackslash 40\% AMI} & \multirow[t]{-3}{*}{\raggedright\arraybackslash bedroom3} & person6 & 544.50 & 0.0719\\
\cmidrule{1-5}
 &  & person4 & 364.83 & 0.1000\\
\cmidrule{3-5}
\multirow[t]{-2}{*}{\raggedright\arraybackslash 50\% AMI} & \multirow[t]{-2}{*}{\raggedright\arraybackslash bedroom3} & person5 & 163.17 & 0.1500\\
\bottomrule
\end{tabu}
\endgroup{}

\newpage

Calculate annual subsidy per household (\texttt{subsidy\_annual}) and
the theoretical share of subsidy allocated for each household
(\texttt{subsidy\_share}). Determine the number of households served
(\texttt{hh\_served}) by normalizing \texttt{subsidy\_share} to the
known budget:

\begin{Shaded}
\begin{Highlighting}[]
\CommentTok{\# Annual subsidy per household type}
\NormalTok{sC\_subsidy\_annual }\OtherTok{\textless{}{-}}\NormalTok{ sC\_redist\_subsidy }\SpecialCharTok{|\textgreater{}}
  \FunctionTok{mutate}\NormalTok{(}\AttributeTok{subsidy\_annual =}\NormalTok{ subsidy }\SpecialCharTok{*} \DecValTok{12}\NormalTok{) }\SpecialCharTok{|\textgreater{}} 
  \FunctionTok{select}\NormalTok{(AMI, bedrooms, hh\_size, }\StringTok{"pct"} \OtherTok{=}\NormalTok{ pct\_redist, subsidy\_annual) }\SpecialCharTok{|\textgreater{}} 
  \FunctionTok{mutate}\NormalTok{(}
    \AttributeTok{subsidy\_share =}\NormalTok{ subsidy\_annual }\SpecialCharTok{*}\NormalTok{ pct, }\CommentTok{\# Subsidy per HH type}
    \AttributeTok{hh\_served =} \CommentTok{\# Adjust to known budget}
\NormalTok{      pct}\SpecialCharTok{*}\NormalTok{(}
\NormalTok{        sC\_budget }\SpecialCharTok{*}\NormalTok{ (}\DecValTok{1} \SpecialCharTok{{-}}\NormalTok{ sC\_overhead)}
\NormalTok{        )}\SpecialCharTok{/}\FunctionTok{sum}\NormalTok{(subsidy\_share) }
\NormalTok{  )}
\end{Highlighting}
\end{Shaded}

\begingroup\fontsize{8}{10}\selectfont

\begin{tabu} to \linewidth {>{\raggedright}X>{\raggedright}X>{\raggedright}X>{\centering}X>{\centering}X>{\centering}X}
\toprule
\textbf{AMI} & \textbf{bedrooms} & \textbf{hh\_size} & \textbf{subsidy\_annual} & \textbf{subsidy\_share} & \textbf{hh\_served}\\
\midrule
 &  & person2 & 9766.50 & 473.06 & 3.257\\
\cmidrule{3-6}
 & \multirow[t]{-2}{*}{\raggedright\arraybackslash bedroom1} & person3 & 7954.50 & 136.72 & 1.156\\
\cmidrule{2-6}
 &  & person3 & 11320.50 & 760.60 & 4.518\\
\cmidrule{3-6}
 & \multirow[t]{-2}{*}{\raggedright\arraybackslash bedroom2} & person4 & 9520.50 & 476.03 & 3.362\\
\cmidrule{2-6}
 &  & person4 & 16434.00 & 25.68 & 0.105\\
\cmidrule{3-6}
 &  & person5 & 14982.00 & 444.78 & 1.996\\
\cmidrule{3-6}
\multirow[t]{-7}{*}[2\dimexpr\aboverulesep+\belowrulesep+\cmidrulewidth]{\raggedright\arraybackslash 30\% AMI} & \multirow[t]{-3}{*}{\raggedright\arraybackslash bedroom3} & person6 & 13530.00 & 486.23 & 2.416\\
\cmidrule{1-6}
 &  & person2 & 4942.50 & 424.75 & 5.778\\
\cmidrule{3-6}
 & \multirow[t]{-2}{*}{\raggedright\arraybackslash bedroom1} & person3 & 2526.50 & 59.21 & 1.576\\
\cmidrule{2-6}
 &  & person3 & 5892.50 & 727.36 & 8.300\\
\cmidrule{3-6}
 & \multirow[t]{-2}{*}{\raggedright\arraybackslash bedroom2} & person4 & 3492.50 & 387.45 & 7.459\\
\cmidrule{2-6}
 &  & person4 & 10406.00 & 32.52 & 0.210\\
\cmidrule{3-6}
 &  & person5 & 8434.00 & 685.26 & 5.463\\
\cmidrule{3-6}
\multirow[t]{-7}{*}[2\dimexpr\aboverulesep+\belowrulesep+\cmidrulewidth]{\raggedright\arraybackslash 40\% AMI} & \multirow[t]{-3}{*}{\raggedright\arraybackslash bedroom3} & person6 & 6534.00 & 469.63 & 4.833\\
\cmidrule{1-6}
 &  & person4 & 4378.00 & 437.80 & 6.724\\
\cmidrule{3-6}
\multirow[t]{-2}{*}{\raggedright\arraybackslash 50\% AMI} & \multirow[t]{-2}{*}{\raggedright\arraybackslash bedroom3} & person5 & 1958.00 & 293.70 & 10.086\\
\bottomrule
\end{tabu}
\endgroup{}

\newpage

Calculate and summarize the estimated budget and households served by
AMI and household size:

\begin{Shaded}
\begin{Highlighting}[]
\NormalTok{sC\_served }\OtherTok{\textless{}{-}}\NormalTok{ sC\_subsidy\_annual }\SpecialCharTok{|\textgreater{}} 
  \FunctionTok{mutate}\NormalTok{(}\AttributeTok{budget =}\NormalTok{ hh\_served }\SpecialCharTok{*}\NormalTok{ subsidy\_annual) }\SpecialCharTok{|\textgreater{}} 
  \FunctionTok{summarise}\NormalTok{(}
    \AttributeTok{budget =} \FunctionTok{sum}\NormalTok{(budget),}
    \AttributeTok{hh\_served =} \FunctionTok{sum}\NormalTok{(hh\_served),}
    \AttributeTok{.by =} \FunctionTok{c}\NormalTok{(AMI, hh\_size)}
\NormalTok{  )}
\end{Highlighting}
\end{Shaded}

\begingroup\fontsize{8}{10}\selectfont

\begin{tabu} to \linewidth {>{\raggedright}X>{\raggedright}X>{\centering}X>{\centering}X}
\toprule
\textbf{AMI} & \textbf{hh\_size} & \textbf{budget} & \textbf{hh\_served}\\
\midrule
 & person2 & 31808.22 & 3.257\\
\cmidrule{2-4}
 & person3 & 60334.15 & 5.673\\
\cmidrule{2-4}
 & person4 & 33733.82 & 3.467\\
\cmidrule{2-4}
 & person5 & 29906.27 & 1.996\\
\cmidrule{2-4}
\multirow[t]{-5}{*}[4\dimexpr\aboverulesep+\belowrulesep+\cmidrulewidth]{\raggedright\arraybackslash 30\% AMI} & person6 & 32693.73 & 2.416\\
\cmidrule{1-4}
 & person2 & 28559.34 & 5.778\\
\cmidrule{2-4}
 & person3 & 52887.90 & 9.876\\
\cmidrule{2-4}
 & person4 & 28238.06 & 7.669\\
\cmidrule{2-4}
 & person5 & 46076.10 & 5.463\\
\cmidrule{2-4}
\multirow[t]{-5}{*}[4\dimexpr\aboverulesep+\belowrulesep+\cmidrulewidth]{\raggedright\arraybackslash 40\% AMI} & person6 & 31577.35 & 4.833\\
\cmidrule{1-4}
 & person4 & 29437.07 & 6.724\\
\cmidrule{2-4}
\multirow[t]{-2}{*}[1\dimexpr\aboverulesep+\belowrulesep+\cmidrulewidth]{\raggedright\arraybackslash 50\% AMI} & person5 & 19747.98 & 10.086\\
\cmidrule{1-4}
Total & - & 425000.00 & 67.239\\
\bottomrule
\end{tabu}
\endgroup{}

\newpage

\subsection{Model results}\label{model-results-2}

Round each estimate to the nearest whole number and determine total:

\begin{Shaded}
\begin{Highlighting}[]
\CommentTok{\# Rounded estimates with grand total}
\NormalTok{sC\_estimate }\OtherTok{\textless{}{-}}\NormalTok{ sC\_served }\SpecialCharTok{|\textgreater{}}
  \FunctionTok{arrange}\NormalTok{(AMI, hh\_size) }\SpecialCharTok{|\textgreater{}} 
  \FunctionTok{mutate}\NormalTok{(}
    \AttributeTok{hh\_served =} \FunctionTok{round}\NormalTok{(hh\_served),}
    \AttributeTok{hh\_size =} \FunctionTok{case\_match}\NormalTok{(}
\NormalTok{      hh\_size,}
      \StringTok{"person2"} \SpecialCharTok{\textasciitilde{}} \StringTok{"2 person"}\NormalTok{,}
      \StringTok{"person3"} \SpecialCharTok{\textasciitilde{}} \StringTok{"3 person"}\NormalTok{,}
      \StringTok{"person4"} \SpecialCharTok{\textasciitilde{}} \StringTok{"4 person"}\NormalTok{,}
      \StringTok{"person5"} \SpecialCharTok{\textasciitilde{}} \StringTok{"5 person"}\NormalTok{,}
      \StringTok{"person6"} \SpecialCharTok{\textasciitilde{}} \StringTok{"6 person"}
\NormalTok{    )}
\NormalTok{  ) }\SpecialCharTok{|\textgreater{}} 
  \FunctionTok{adorn\_totals}\NormalTok{()}
\end{Highlighting}
\end{Shaded}

\begin{table}[H]

\caption{\label{tbl-sC}Scenario C - Estimated Households Served by
Household Size}

\centering{

\begin{tabu} to \linewidth {>{\raggedright}X>{\raggedright}X>{\centering}X>{\centering}X}
\toprule
\textbf{Income} & \textbf{Household size} & \textbf{Budget} & \textbf{Households served}\\
\midrule
 & 2 person & \$31,808 & 3\\
\cmidrule{2-4}
 & 3 person & \$60,334 & 6\\
\cmidrule{2-4}
 & 4 person & \$33,734 & 3\\
\cmidrule{2-4}
 & 5 person & \$29,906 & 2\\
\cmidrule{2-4}
\multirow[t]{-5}{*}[4\dimexpr\aboverulesep+\belowrulesep+\cmidrulewidth]{\raggedright\arraybackslash 30\% AMI} & 6 person & \$32,694 & 2\\
\cmidrule{1-4}
 & 2 person & \$28,559 & 6\\
\cmidrule{2-4}
 & 3 person & \$52,888 & 10\\
\cmidrule{2-4}
 & 4 person & \$28,238 & 8\\
\cmidrule{2-4}
 & 5 person & \$46,076 & 5\\
\cmidrule{2-4}
\multirow[t]{-5}{*}[4\dimexpr\aboverulesep+\belowrulesep+\cmidrulewidth]{\raggedright\arraybackslash 40\% AMI} & 6 person & \$31,577 & 5\\
\cmidrule{1-4}
 & 4 person & \$29,437 & 7\\
\cmidrule{2-4}
\multirow[t]{-2}{*}[1\dimexpr\aboverulesep+\belowrulesep+\cmidrulewidth]{\raggedright\arraybackslash 50\% AMI} & 5 person & \$19,748 & 10\\
\cmidrule{1-4}
\textbf{Total} & \textbf{-} & \textbf{\$425,000} & \textbf{67}\\
\bottomrule
\end{tabu}

}

\end{table}%

\hfill\break

\begin{tcolorbox}[enhanced jigsaw, breakable, toprule=.15mm, opacitybacktitle=0.6, colframe=quarto-callout-note-color-frame, colback=white, opacityback=0, coltitle=black, colbacktitle=quarto-callout-note-color!10!white, rightrule=.15mm, toptitle=1mm, titlerule=0mm, title=\textcolor{quarto-callout-note-color}{\faInfo}\hspace{0.5em}{Scenario C results}, bottomtitle=1mm, bottomrule=.15mm, leftrule=.75mm, arc=.35mm, left=2mm]

\emph{Average annual program cost per household: \$7,462.69}

\hfill\break
Under Scenario C, a total program budget of \$500,000 with a 15\%
administrative overhead leaves \$425,000 to fund rental assistance.
Given the assumed household distributions by AMI, unit size, and
household size, the total number of households served is 67.

\end{tcolorbox}



\end{document}
